\documentclass[a4paper]{scrartcl}
% this template is an adaption of:
% https://www.overleaf.com/latex/templates/template-for-theoretische-informatik-uni-tubingen/xwsycshfkjtf


% change for each exercise
\newcommand{\NUMBER}{1} % assignment number
\newcommand{\EXERCISES}{3} % number of excercises

% only once
\usepackage{makecell}
\newcommand{\COURSE}{Lecture Name, Winter Semester 23/24}
\newcommand{\STUDENTA}{\makecell{Koray Ulusan\\ID: 0000000}}
\newcommand{\STUDENTB}{\makecell{Lorem Ipsum\\ID: ???????}}

% --------------------------------------------------------
\usepackage[utf8]{inputenc}
\usepackage[english]{babel}
\usepackage{amsmath}
\usepackage{amssymb}
\usepackage{fancyhdr}
\usepackage{color}
\usepackage{graphicx}
\usepackage{lastpage}
\usepackage{listings}
\usepackage{tikz}
\usepackage{pdflscape}
\usepackage{subfigure}
\usepackage{float}
\usepackage{polynom}
% \usepackage{hyperref}
\usepackage{tabularx}
\usepackage{forloop}
\usepackage{geometry}
\usepackage{listings}
\usepackage{fancybox}
\usepackage{tikz}
\usepackage{algpseudocode,algorithm,algorithmicx}


\input kvmacros
\geometry{a4paper,left=3cm, right=3cm, top=3cm, bottom=3cm}

% headline
\def\header#1{
	\begin{center}
		{\Large Assignment Sheet #1}
	\end{center}
}

% point table
\newcounter{punktelistectr}
\newcounter{punkte}
\newcommand{\punkteliste}[2]{%
	\setcounter{punkte}{#2}%
	\addtocounter{punkte}{-#1}%
	\stepcounter{punkte}%<-- also punkte = m-n+1 = Anzahl Spalten[1]
	\begin{center}%
		\begin{tabularx}{\linewidth}[]{@{}*{\thepunkte}{>{\centering\arraybackslash} X|}@{}>{\centering\arraybackslash}X}
			\forloop{punktelistectr}{#1}{\value{punktelistectr} < #2 } %
			{%
				\thepunktelistectr &
			}
			#2 &  $\Sigma$ \\
			\hline
			\forloop{punktelistectr}{#1}{\value{punktelistectr} < #2 } %
			{%
				&
			} &\\
			\forloop{punktelistectr}{#1}{\value{punktelistectr} < #2 } %
			{%
				&
			} &\\
		\end{tabularx}
	\end{center}
}

% header / footer
\pagestyle {fancy}
\fancyhead[L]{\COURSE}
\fancyhead[C]{}
\fancyhead[R]{\today}

\fancyfoot[L]{}
\fancyfoot[C]{}
\fancyfoot[R]{Page \thepage /\pageref*{LastPage}}
\setlength{\headheight}{23.6pt}

% style:
\usepackage{../korays-homework-sty/korayHomework}
\renewcommand{\questionPreText}{Question }
\renewcommand{\subquestionPreText}{Part }
\renewcommand{\subsubquestionPreText}{}
\renewcommand{\subquestionPostText}{)}

% custom additions:
\usepackage{ wasysym }
\usepackage{enumerate}% http://ctan.org/pkg/enumerate
\usepackage{dsfont}

\usepackage{tikz}
\usetikzlibrary{automata, arrows.meta, positioning}

\def\ddfrac#1#2{\displaystyle\frac{\displaystyle #1}{\displaystyle #2}}

\begin{document}

% names and grades
\begin{tabularx}{\linewidth}{m{0.3 \linewidth}X}
	\begin{minipage}{\linewidth}
		\tcbset{
			colframe=contourColor,colback=themeColor,
			top=0.35em,bottom=0.2em,
			left=-0.40em,right=0.4em
		}
		\tcbox[tikznode]{
			\begin{tabular}{ll}
				% student A
				\raisebox{-.37\height}{
					\begin{tikzpicture}
						\clip (0,0) circle (0.5cm) node {\includegraphics[width=1cm]{../korays-homework-sty/img/koray-min-pfp.jpg}};
					\end{tikzpicture}
				} & \hspace{-1em} \STUDENTA \\
				% student B
				${}$\vspace{-0.7em}         \\ % adjust vertical space between rows
				\raisebox{-.37\height}{
					\begin{tikzpicture}
						\clip (0,0) circle (0.5cm) node {\includegraphics[width=1cm]{../korays-homework-sty/img/unknown2.jpg}};
					\end{tikzpicture}
				} & \hspace{-1em} \STUDENTB \\
			\end{tabular}
		}
	\end{minipage} & \begin{minipage}{\linewidth}
		                 \punkteliste{1}{\EXERCISES}
	                 \end{minipage} \\
\end{tabularx}

\vspace*{-3em} % dirty hack
\header{\NUMBER}
% -------------------------------------------------------------------

\vspace*{-1.3em}

\question{1}
\subquestion{a}
\subquestion{b}
\subquestion{c}
\question{2}
\subquestion{a}
\subsubquestion{i}
\subsubquestion{ii}
\subquestion{b}
\question{3}
\subquestion{a}
\subquestion{b}

\vfill
\ornamento
\KoraysHomeworkStyCredit

\end{document}
